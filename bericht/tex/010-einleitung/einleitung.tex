\section{Einleitung}

Das Semesterprojekt „Kommunizierende Systeme” wurde vom Lehrstuhl der Technischen Informatik an der
Humboldt-Universität zu Berlin organisiert und umfasste 12 studentische Teilnehmer.
Betreut wurde es von Stefan Dietzel, Phillipp Schoppmann, Sebastian Henningsen und Björn
Scheuermann.
Das Projekt begann mit dem Wintersemester im Oktober 2017.
Die Zwischenpräsentation fand am 19. Dezember 2017 statt und die Endpräsentation war am
16. Februar 2018.

Das Ziel des Projekts ist es die Studenten zuvor erworbene Kenntnisse aus
Software Engineering und Grundlagen der Programmierung praktisch anwenden zu
lassen. Zur Lösung der gegeben Aufgabe sollen sie ein komplexes System
entwerfen, implementieren, testen und dokumentieren.

Die Aufgabe war die Umsetzung eines eigenen Lasertag mit modifizierten Regeln.
Bei Lasertag handelt es sich um ein Spiel, bei dem mehrere Spieler mit Sendern
in Form von Pistolen und Sensoren ausgestattet werden. Ziel ist es andere
Spieler abzuschießen, um Punkte zu sammeln. Hierbei gibt es verschiedene
Spielmodi, z.B. teambasiert oder jeder gegen jeden. Die besondere Herausforderung
war, dass die Spielregeln so angepasst werden sollten, dass sie einem Peer-to-Peer
Netzwerkprotokoll entsprechen, welches von den Studenten gewählt werden konnten.
Der Entwurf des Systems, dass diese Aufgabe löst, wurde den Studenten völlig
überlassen. Die Spielausrüstung sollte eine physische Form annehmen, sodass
sich die Spieler durch den Raum bewegen können und mit ihrer Umgebung
interagieren können. Es war den Teilnehmern erlaubt Kompromisse zwischen den
originalen Netzwerkprotokollen und der Spielbarkeit zu machen.

Für die Organisation des Projektes wurde Slack verwendet.
Außerdem wurde der vom Informatik-Institut zur Verfügung stehende Gitlab-Server für die
Versionskontrolle genutzt.
Beides wurde durch die Betreuer des Projektes vorgegeben.
