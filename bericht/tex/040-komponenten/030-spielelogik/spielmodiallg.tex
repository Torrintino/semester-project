\subsubsection{Spielmodi allgemein}
Die wichtigste Anforderung seitens des Kunden für die Umsetzung der Spielmodi war, dass eines oder mehrere Ad-Hoc Multi-Hop Netzwerkprotokolle diesen zugrundeliegen sollten. Dabei sollen die Entscheidungen für Interaktionen von den Spielenden möglichst selbst getroffen werden, damit die Spielenden lernen können, wie das jeweils gespielte Netzwerkprotokoll funktioniert. Darüberhinaus war dem Kunden jedoch auch die Spielbarkeit (Fairness, Spannung, Unterhaltung) wichtig. Als Vorbild in diesem Zusammenhang sollte Lasertag dienen. Es war dem Kunden und auch uns dabei klar, dass das Modifikationen am jeweiligen Netzwerkprotokoll erforderlich machen würde.

Als Ideen für die Umsetzung dieser Anforderungen standen zwei grobe Szenarien zur Diskussion: ein Protokoll aus dem Bereich der Wireless Sensor Networks oder der Peer-to-Peer Content Distribution. Nach eingehender Beschäftigung mit einigen Protokollen aus dem Bereich der Wireless Sensor Networks gab es erhebliche Zweifel an der Umsetzbarkeit eines Spiels auf deren Basis unter Berücksichtigung der genannten Spielbarkeitskriterien Spannung und Unterhaltung und nach dem Vorbild von Lasertag. So besteht ein Wireless Sensor Network meist aus vielen Wireless Sensor Nodes, die gut von den Spielenden repräsentiert hätten werden können. Jedoch arbeiten diese Knoten in der Praxis oft zusammen an der Erfüllung eines gemeinsamen Ziels. Wir haben keinen Konkurrenzgedanken, keine egoistischen Motive der einzelnen Netzwerkknoten in Wireless Sensor Networks gesehen, die sich gut zur Umsetzung eines Spiels nach dem Vorbild von Lasertag geeignet hätten. Stattdessen gibt es Ziele in Wireless Sensor Networks wie die Energiesparsamkeit der einzelnen Geräte, deren Umsetzung für uns im Widerspruch zu einem action-geladenen Spiel wie Lasertag standen. Das Spiel auf dieser Basis wäre eher statisch geworden.

Nach Beschäftigung mit Peer-to-Peer Content Distribution erschien diese uns deutlich besser als Grundlage für das Spiel geeignet. Wir haben uns dabei für den bekanntesten Vertreter seiner Art, BitTorrent, entschieden, denn wir haben folgende Analogien zwischen dessen Funktionsweise und Lasertag gefunden, die sich durch alle von uns erstellten Spielmodi ziehen:

\begin{enumerate}
\item Bei Lasertag interagieren die Spielenden miteinander, indem sie mit ihrer Spielwaffe auf die Weste eines anderen Spielenden schießen. Diese Interaktion, der Schuss, entspricht in unserem Spiel im Sinne von BitTorrent der Initiierung einer Datenübertragung zwischen den beiden Spielenden. Der Schuss zwingt den Getroffenen im Allgemeinen zur Übertragung eines Dateifragments an den Schützen. Dabei wird der Einfachheit halber automatisch das erste Dateifragment ausgewählt, das der Schütze noch nicht hat. Sollte es kein solches Dateifragment geben, dass der Getroffene besitzt und der Schütze noch nicht, so ändert sich entsprechend nichts an den Dateifragmenten beider Spieler.

Anmerkung: Da es sich um ein Spiel handelt, wird nicht wirklich ein Dateifragment übertragen, sondern lediglich die Information über dessen (theoretischen) Besitz.

\item Bei Lasertag hat jeder Spielende standardmäßig das (egoistische) Ziel, möglichst viele seiner Gegner zu treffen bzw. abzuschießen, und er bekommt dafür Punkte. Bei BitTorrent hat jeder Teilnehmer im Netzwerk das (ebenfalls egoistische) Ziel, möglichst schnell die gewünschte Datei(en) von seinen Peers vollständig zu erhalten. Entsprechend vergeben wir im Allgemeinen (wenn nicht anders beim jeweiligen Spielmodus erklärt) auch in unserem Spiel Punkte für erfolgreiche und für das Erreichen des egoistischen Ziels sinnvolle Interaktionen, also Datenübertragungen zwischen den Peers.

\item Bei Lasertag gibt es das Konzept der Unverwundbarkeitszeit. Wenn ein Spielender einen anderen Spielenden getroffen hat, dann ist der Getroffene üblicherweise für eine gewisse Zeit (einige Sekunden) vor weiteren Schüssen auf ihn geschützt. Das verhindert unfairen Mehrfachbeschuss von einem Spieler innerhalb kurzer Zeit. Diese Zeit kann der Getroffene nutzen, um sich selbst wieder in Deckung zu begeben.

Wir haben dieses Konzept im Prinzip aus denselben Gründen in unsere Spielmodi übernommen. Im Kontext von BitTorrent lässt sich diese Unverwundbarkeitszeit jedoch auch einfach als die Zeit erklären, die die durch den Schuss initiierte Datenübertragung (für ein Dateifragment) benötigen würde. In unserem Spiel kann dieser Interpretation folgend jeder Schütze zwar mehrere Dateifragmente gleichzeitig erhalten, jedoch kein Getroffener mehr als ein Dateifragment gleichzeitig an einen oder mehrere Schützen übertragen. Natürlich wären in der Realität bei BitTorrent mehrere Datenübertragungen in beide Richtungen gleichzeitig möglich und sogar anzustreben. Die genannte Interpretion ist also gleichzeitig eine der Vereinfachungen, die wir zur besseren Spielbarkeit vorgenommen haben.

\item Bei Lasertag gibt es Spielmodi in Teams. Dieses Spiel in Teams lässt sich gut zur Darstellung des kollaborativen Ansatzes von BitTorrent nutzen. Denn trotz der egoistischen Motive der Peers im Netzwerk braucht jeder Peer die anderen Peers und trägt entsprechend auch selbst etwas zur Erreichung der Ziele anderer Peers bei, indem er ihnen bereits erhaltene Dateifragmente zur Verfügung stellt.
\end{enumerate}

\noindent
Allgemein folgen all unsere Spielmodi den folgenden Annahmen. Diese sind größtenteils Vereinfachungen, die die Komplexität des Spiels reduzieren und es damit leichter spielbar machen.

\begin{enumerate}
\item Jeder Spielende besitzt zum Start ein Dateifragment, welches keiner seiner “Peers” (alle anderen Spielenden) besitzt.
\item Es geht insgesamt um eine Datei, die in entsprechend viele Dateifragmente zerlegt ist, wie es Spielende im Spiel gibt.
\item Ziel ist allgemein, bis auf Ausnahmen, die beim entsprechenden Spielmodus beschrieben sind, die eine Datei als Erster vollständig von den Peers zu erhalten.
\end{enumerate}

\noindent
Die initiale Verteilung der Dateifragmente auf die Peers sorgt für einen schnellen und relativ einfachen Einstieg in das jeweilige Spiel, weil sich so zu Beginn ein Schuss auf jeden Peer lohnt. Entsprechend nimmt die Komplexität des Spiels für jeden Spielenden mit der Zeit zu. So muss jeder Spielende zunehmend im Auge behalten, welche Dateifragmente er schon hat und von welchem Peer er die fehlenden bekommen kann.

Wir haben verschiedene Spielmodi entworfen, die grundsätzlich auf den genannten Analogien und Annahmen beruhen. Sie unterscheiden sich teilweise in der Komplexität ihrer Regeln und haben teilweise verschiedene Szenarien als Ausgangspunkt, die entsprechenden Einfluss auf das Spiel und dessen Regeln nehmen.