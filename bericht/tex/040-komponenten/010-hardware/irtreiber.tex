\subsubsection{IR-Treiber}

Um die einzelnen Spieler zu unterscheiden, war es notwendig, dass sie unterschiedliche
Informationen aussenden, um sie eindeutig im Spielkontext zu identifizieren.
Wir hatten 3 verschiedene Ideen, wie man diese Informationen modellieren kann, um sie schnell und 
korrekt zu übertragen.
Dabei war die Grundlage, dass jede gesendete Information in eine eindeutige Abfolge von 2
verschiedenen Zuständen, im Folgenden als Zustand 0 (kurz: 0) bzw. Zustand 1 (kurz: 1)
bezeichnet, übersetzt wird und dann diese Abfolge von Zuständen gesendet wird.
Diese Abfolge von Zuständen musste ebenfalls modelliert werden, um das Gebot der Korrektheit zu
erfüllen, da ein einfaches Absenden der Information mit Signal an $= 1$ und Signal aus $= 0$ sehr
rauschanfällig wäre.
Des Weiteren wollten wir auf selbstkorrigierende Codes verzichten, um eine möglichst schnelle und
genaue Treffererkennung zu schaffen, und um zu verhindern, dass wenn 2 Spieler nahezu gleichzeitig
schießen, der Spieler, der zuerst schießt, den Treffer nicht gewertet bekommt, da sein Schuss
erst korrigiert werden musste, wohingegen der Schuss des langsameren Spielers zuerst gewertet
wird, da sein Schuss keine Korrektur hatte.

Dazu hatten wir 3 Ideen, wie man diese Modulation umsetzen könnte:
\begin{enumerate}
  \item
    Man gibt Zustand 1 und Zustand 0 unterschiedliche Längen von einem Signal mit festen Pausen
    dazwischen.
  \item
	Man modeliert die Zustände über Signalflanken.
  \item
	Man gibt beiden Zuständen die gleiche Signalzeit, aber unterschiedliche Pausen zwischen den
	einzelnen Signalen.
\end{enumerate}

\paragraph{Idee 1}
Der Vorteil der ersten Idee ist, dass sie sehr simpel zu implementieren ist.

Der Nachteil ist jedoch, dass sie von den 3 Ideen am stärksten rauschanfällig ist und somit die
geringste Reichweite bietet.

\paragraph{Idee 2}
Die Vorteile hier bestehen darin, dass sie sehr einfach zu implementieren wäre, wenn man den Treiber
von Grund auf selbst schreibt, und dass sie die schnellste Übertragung von allen bietet.

Der größte Nachteil besteht jedoch darin, dass diese Idee selber sehr rauschanfällig ist, da Flanken
in der Theorie überspielt werden könnten.

\paragraph{Idee 3}
Der Vorteil besteht hier darin, dass die wenigste Rauschanfälligkeit besteht, da man – mit genügend
großen Unterschieden zwischen den Pausen bei den beiden Zuständen – sehr großzügig sein kann, was
noch als Treffer gilt.
Ein weiterer Vorteil war, dass man bereits Geräte im Haus hatte, die genauso funktionieren, was
Tests am Anfang stark vereinfacht hatte und sie dafür gesorgt hat, dass das Troubleshooting später
ebenfalls leichter war.

An Nachteilen ist hier aufzulisten, dass es von den 3 Ideen am schwierigsten zu implementieren
gewesen wäre und dass es die langsamste Variante für die Informationsübertragung ist. \\

Am Ende hatten wir uns für Idee 3 entschieden, da die Informationspakete klein genug sind, dass
die Geschwindigkeit nicht leidet.
Außerdem haben wir das Problem mit der Implementation dadurch umgangen, dass man den
Open-Source-Treiber Lirc benutzt hat, sodass man nicht den Treiber selber schreiben musste, sondern
nur über die Lirc-bibliothek mit dem Treiber kommunizieren musste.
Der Vorteil eines externen Testgerätes war dann den Vorteilen der anderen beiden Varianten stark
überlegen.
