\subsubsection{IR-Treiber}

Um die einzelnen Spieler zu unterscheiden, war es notwendig, dass sie unterschiedliche
Informationen aussenden, um sie eindeutig im Spielkontext zu identifizieren.
Wir hatten drei verschiedene Ideen, wie man diese Informationen modellieren kann, um sie schnell und 
korrekt zu übertragen.
Dabei war die Grundlage, dass jede gesendete Information in eine eindeutige Abfolge von 2
verschiedenen Zuständen – im Folgenden als Zustand 0 (kurz: 0) bzw. Zustand 1 (kurz: 1)
bezeichnet – übersetzt wird und dann diese Abfolge von Zuständen gesendet wird.
Diese Abfolge von Zuständen musste ebenfalls modelliert werden, um das Gebot der Korrektheit zu
erfüllen, da ein einfaches Absenden der Information – mit Signal an $= 1$ und Signal aus $= 0$ –
sehr rauschanfällig wäre.
Des Weiteren wollten wir auf selbstkorrigierende Codes verzichten, um eine möglichst schnelle und
genaue Treffererkennung zu schaffen, und um zu verhindern, dass wenn zwei Spieler nahezu
gleichzeitig schießen, der Spieler, der zuerst schießt, den Treffer nicht gewertet bekommt, da sein
Schuss erst korrigiert werden musste, wohingegen der Schuss des langsameren Spielers zuerst gewertet
wird, da sein Schuss keine Korrektur hatte.
Dazu hatten wir folgende drei Ideen, wie man diese Modulation umsetzen könnte:
\begin{enumerate}
  \item
    Man gibt Zustand 1 und Zustand 0 unterschiedliche Längen von einem Signal mit festen Pausen
    dazwischen.
  \item
	Man modeliert die Zustände über Signalflanken.
  \item
	Man gibt beiden Zuständen die gleiche Signalzeit, aber unterschiedliche Pausen zwischen den
	einzelnen Signalen.
\end{enumerate}

\paragraph{Idee 1}
Der Vorteil der ersten Idee ist, dass sie sehr simpel zu implementieren ist.
Der Nachteil ist jedoch, dass sie von den drei Ideen am stärksten rauschanfällig ist und somit die
geringste Reichweite bietet.

\paragraph{Idee 2}
Die Vorteile hier bestehen darin, dass sie sehr einfach zu implementieren wäre, wenn man den Treiber
von Grund auf selbst schreibt, und dass sie die schnellste Übertragung von allen bietet.
Der größte Nachteil besteht jedoch darin, dass diese Idee selbst sehr rauschanfällig ist, da Flanken
in der Theorie überspielt werden könnten.

\paragraph{Idee 3}
Der Vorteil besteht hier darin, dass die geringste Rauschanfälligkeit besteht, da man – mit genügend
großen Unterschieden zwischen den Pausen bei den beiden Zuständen – sehr großzügig sein kann, was
noch als Treffer gilt.
Ein weiterer Vorteil war, dass man bereits Geräte im Haus hatte, die genauso funktionieren, was
Tests am Anfang stark vereinfacht hatte und dafür gesorgt hat, dass eine Fehlerbehebung später
ebenfalls leichter war.
An Nachteilen ist hier aufzulisten, dass es von den drei Ideen am schwierigsten zu implementieren
gewesen wäre und dass es die langsamste Variante für die Informationsübertragung ist. \\

Am Ende hatten wir uns für Idee 3 entschieden, da die Informationspakete klein genug sind, dass
die Geschwindigkeit nicht leidet.
Außerdem haben wir das Problem mit der Implementation dadurch umgangen, dass man den
Open-Source-Treiber Lirc benutzt hat, sodass man nicht den Treiber selber schreiben musste, sondern
nur über die Lirc-Bibliothek mit dem Treiber kommunizieren musste.
Der Vorteil eines externen Testgerätes war dann den Vorteilen der anderen beiden Varianten stark
überlegen.

\paragraph{Kommunikation mit Lirc}
Kommunikation mit Lirc



Lirc lädt sobald man es startet, ein File aus dem es ausliest wie es konfiguriert ist.

Hier ein Beispiel-file:


begin remote

  name  12		   	
  bits           15  		
  flags SPACE_ENC			 	
  eps            25
  aeps          100

  one           364  1739
  zero          364   690
  ptrail        364

      begin codes
          KEY_0                    0x00000000000043A2        
          KEY_1                    0x0000000000004202        
          KEY_2                    0x0000000000004102        
          KEY_3                    0x0000000000004302        
          KEY_4                    0x0000000000004082       
          KEY_5                    0x0000000000004282        
          KEY_6                    0x0000000000004182        
      end codes

end remote

Der Name der Fernbedienung ist wichtig, falls man mehrere Fernbedienungen im Einsatz hat. Da wir uns aber entschieden haben,
dass man jeden Tagger einfach als einen Knopf darzustellen, da dass am einfachsten einzustellen ist und keinen Nachteil
gegenüber den anderen Methoden hat, ist der Name hier egal.

“bits“ gibt an, wie viele verschiedene Bits verschickt werden.



“Flags“ gibt an, welche Modulationsvariante man benutzt, in dem File hier wird die gleiche Variante angegeben,
die im Projekt selbst benutzt wird.



“eps“ gibt an, wie viel relativer Fehler in einem Code auftauchen dürfen, damit ein Code noch als korrkt erkannt wird.



“aeps“ gibt an, wie viele absolute Fehler pro Bit erlaubt sind.

“one“ und “zero“ geben an, wie eine 1 bzw. eine 0 simuliert werden.
Die erste Zahl gibt die Sendezeit in Millisekunden an, die zweite Zahl sagt aus, wie viele Millisekunden gewartet wird,
bis das nächste Bit gesendet wird.



“ptrail“ gibt an ob und wie lange gesendet werden soll, nachdem das letzte Bit eines Codes gesendet wurde. 
Dies macht eine 1 von einer 0 auf dem letztem Bit unterscheidbar.



In dem Bereich nach “begin codes“ steht, welcher Code zu welcher Taste gehört. Dabei haben wir die Tagger-ID mit dem
entsprechendem Key verbunden, also hat der Tagger mit der ID “1“, den KEY_1 Code zugeordnet. Empfängt also
jemand KEY_1, wird davon ausgegangen, dass er von dem Tagger mit der ID “1“ getroffen wurde.



Lirc stellt einem alle Empfangenem Codes auf einem Socket zur Verfügung. Diesen bekommt man beim initialisieren mit der
 „lirc_init()“ Funktion mitgeteilt.


Diese kann man mit der Funktion „lirc_nextcode()“ auslesen und verarbeiten. Da wir nur eine Fernbedienung haben
und, durch das ID-Matching,nur die Key-Nummer wichtig ist, wird diese an der entsprechenden Stelle ausgelesen und weitergegeben.



Das Senden der Ids wird mit der „lirc_send_one()“ Funktion realisiert. Hier wird nur die ID des Taggers angegeben
und welche Fernbedienung benutzt wird.


Da viele Sachen normiert sind, da wir viele Funktionalitäten von Lirc nicht nutzen müssen, haben wir Funktionen geschrieben,
die alle festgesetzten Werte automatisch eintragen, sodass die Hardware-API nur die ID des Taggers weitergeben muss,
wodurch wir Fehler minimiert haben. Diese Menge an Funktionen wurde von uns intern als “Treiber“ bezeichnet.
