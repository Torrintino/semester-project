\subsubsection{Betriebssysteminstallation}

Dieser Abschnitt erläutert, wie die Grundinstallation auf den Spielgeräten und dem Server
vorgenommen wird.
Für diese Aufgabe war Lennart Weiß verantwortlich.

Der Server wurde auf einem Raspberry Pi Model B+ eingerichtet. Strenggenommen
sind für den Server nur Strom und ein Wifi Netzwerk absolut erforderlich. Per
Wifi kann der Server als Access Point dienen und der Administrator kann sich mit
SSH auf dem Gerät einloggen und Änderungen vornehmen. Das gewählte Gerät bringt
allerdings noch einige Funktionen mit, welche die Wartung erleichtern. Der
zusätzliche Ethernetanschluss wird genutzt, um ihn mit einem Netzwerk zu
verbinden, dass ihm eine DHCP Adresse zuweist und Internetzugang gewährt. 
Dadurch können der Server, sowie alle mit dem Wifi Netzwerk verbundenen Geräte
Pakete aus dem Internet herunterladen. Sollte aus etwaigen Gründen das Wifi
Netzwerk abstürzen, kann man den Server über HDMI an einen Bildschirm
anschließen und mit einer USB Tastatur warten. Sollte ein Bootvorgang nicht
mehr möglich sein, kann man einfach die SD Karte entfernen und auf einem anderen
Gerät das Dateisystem überprüfen und reparieren. Natürlich hätte man als Server
auch einen gewöhnlichen Desktop PC verwenden können, dieser lässt sich jedoch
nicht so leicht transportieren. Der kleine und leichte Raspberry Pi eignet sich
also besser für den Fall, dass es kein festgelegtes Spielfeld gibt.

Auf dem Server ist Raspbian Lite installiert. Raspbian wurde vor allem wegen 
seiner weiten Verbreitung und Stabilität gewählt. Für Probleme bei der 
Serverinstallation kann im Regelfall eine Suche in Debianforen und ähnlichem
eine Lösung gefunden und auf Raspbian übertragen werden. Außerdem ist das
Angebot an Paketen sehr umfangreich. Die Lite Version wurde gewählt, weil eine
grafische Oberfläche für den Betrieb nicht notwendig ist und die gesamte
Installation und Wartung des Geräts über die Kommandozeile erledigt wird.
Durch den Verzicht auf die grafische Oberfläche und weitere unnötige Pakete
werden Ressourcen gespart und es gibt weniger Angriffsfläche für Fehlfunktionen.

Auf dem Server ist ein Puppetmaster eingerichtet.
Damit kann genau festgelegt werden, welche Software auf die Clients verteilt werden soll und die
Installation kann automatisiert werden.
Für jeden Client muss auf dem Server ein Zertifikat signiert werden, dieser manuelle Schritt wird
jedoch nur einmal bei der Erstinstallation des Client vorgenommen.
Für die möglichst schnelle Installation der Spielgeräte wurde ein Image mit Raspbian vorbereitet, in
dem eine deutsche Lokalisierung mit englischer Sprache eingestellt wurde und ein Puppet Client
installiert ist.
Außerdem wurde die WLAN SSID und das Passwort eingerichtet und der Client verbindet sich nach dem
Boot Vorgang automatisch per wpa\_supplicant mit dem Server.
Um den Kopiervorgang zu beschleunigen, wurde das Image auf die minimale Größe (ca. 2G) gehalten.
Dadurch kann das Image in wenigen Minuten auf eine SD Karte geschrieben und in den neuen Raspberry
Pi eingesetzt werden.
Außerdem kann die SD Karte eine beliebige Größe haben.

Um den Installationsvorgang zu beenden, muss auf man sich per SSH auf dem Client einloggen.
Per raspi-config wird das Filesystem auf die maximale Größe eingestellt und der Hostname
eingestellt.
Dieser ist ‚\texttt{client-x}‘, wobei \texttt{x} die Client ID bezeichnet, welche vorher eindeutig
an jedes Gerät vergeben wird.
Die Geräte sind entsprechend beschriftet, damit sie identifiziert werden können.
Schließlich werden mit Puppet alle benötigten Komponenten installiert und der Client ist
einsatzbereit.

