\subsubsection{Betriebssysteminstallation}
\label{sec:betriebssysteminstallation}

Dieser Abschnitt erläutert, wie die Grundinstallation auf den Spielgeräten und dem Server
vorgenommen wird.
Für diese Aufgabe war Lennart Weiß verantwortlich.

Der Server wurde auf einem Raspberry Pi Model B+ eingerichtet. Strenggenommen
sind für den Server nur Strom und eine Wifi-Netzwerkschnittstelle absolut erforderlich. Per
Wifi kann der Server als Access Point dienen und der Administrator kann sich mit
SSH auf dem Gerät einloggen und Änderungen vornehmen. Das gewählte Gerät bringt
allerdings noch einige Funktionen mit, welche die Wartung erleichtern. Der
zusätzliche Ethernetanschluss wird genutzt, um den Pi mit einem Netzwerk zu
verbinden, das ihm eine DHCP-Adresse zuweist und Internetzugang gewährt. 
Dadurch können der Server sowie alle mit dem Wifi-Netzwerk verbundenen Geräte
Pakete aus dem Internet herunterladen. Sollte aus etwaigen Gründen das Wifi-Netzwerk
abstürzen, kann man den Server über HDMI an einen Bildschirm
anschließen und mit einer USB-Tastatur warten. Sollte ein Bootvorgang nicht
mehr möglich sein, kann man einfach die SD-Karte entfernen und auf einem anderen
Gerät das Dateisystem überprüfen und reparieren. Natürlich hätte man als Server
auch einen gewöhnlichen Desktop-PC verwenden können, dieser lässt sich jedoch
nicht so leicht transportieren. Der kleine und leichte Raspberry Pi eignet sich
also besser für den Fall, dass es kein festgelegtes Spielfeld gibt.

Auf dem Server ist Raspbian Lite installiert. Raspbian wurde vor allem wegen 
seiner weiten Verbreitung und Stabilität gewählt. Für Probleme bei der 
Serverinstallation kann im Regelfall durch eine Suche in Debianforen und ähnlichem
eine Lösung gefunden und auf Raspbian übertragen werden. Außerdem ist das
Angebot an Paketen sehr umfangreich. Die Lite-Version wurde gewählt, weil eine
grafische Oberfläche für den Betrieb nicht notwendig ist und die gesamte
Installation und Wartung des Geräts über die Kommandozeile erledigt wird.
Durch den Verzicht auf die grafische Oberfläche und weitere unnötige Pakete
werden Ressourcen gespart und es gibt weniger Angriffsfläche für Fehlfunktionen.

Für die Clientgeräte gibt es einen geregelten Installationsprozess. Dies ist
notwendig, damit die Geräte alle auf demselben Stand sind. Ist dies nicht
gegeben, ist die Fehlersuche bei unerwartetem Verhalten ungleich schwerer. Die
Installation sollte außerdem automatisiert sein, um die Arbeitszeit für das
Einrichten neuer Spielgeräte möglichst zu reduzieren und etwaige Fehler durch
manuelle Arbeit an den Geräten auszuschließen.

Der Raspberry Pi Zero W wurde als Standardgerät festgelegt. Für die Installation
wurde ein Image erstellt, welches eine Minimalbasis darstellt, die den
restlichen Prozess unterstützt. Hierbei handelt es sich um Raspbian Lite, auf dem
die Lokalisierung auf Deutschland mit englischer Sprache eingestellt wurde, um
die Wartung zu erleichtern. Das Wifi ist so konfiguriert, dass sich der Client
beim Systemstart automatisch mit \texttt{wpa\_supplicant} zum Coding-Tag-Netzwerk
verbindet. Außerdem wurde SSH aktiviert, somit ist es möglich, sich auf dem Gerät
einzuloggen. Der erste Schritt besteht also darin, das Image auf die neue
SD-Karte zu schreiben. Der Client bekommt eine noch nicht vergebene ID, welche
auf das SD-Kartengehäuse mit einem wasserfesten Stift geschrieben wird, damit
das Gerät auch von außen sofort identifizierbar ist. Das Gerät wird als nächstes
mit Strom verbunden und es verbindet sich zum Server. Per SSH loggt man sich
nun ein, dabei muss der Standardhostname ‚Client‘ verwendet werden. Ist dies
getan, wird als erstes der Hostname zu „\texttt{client-x}“ geändert, wobei 
\texttt{x} die Client ID bezeichnet. Das Image wurde auf eine minimale Größe
von etwa 2GB gehalten, um den Schreibprozess zu beschleunigen und
keine konkrete SD-Kartengröße festzulegen. Daher muss die Größe des Dateisystems
nach dem Kopiervorgang wieder maximiert werden. Die meisten zuvor genannten
Schritte lassen sich einfach mit \texttt{raspi-config} vornehmen. Für die
Installation der Projektkomponenten wird wie im \cref{sec:installation-des-projekts}
beschrieben Puppet verwendet. Puppet ist bereits im Image installiert, aber
als zusätzlicher Schritt muss für den Client ein neues Zertifikat erstellt
werden. Dafür verbindet sich der Puppet Client einmal mit dem Server und das
neu erstellte Zertifikat wird auf dem Server signiert.

