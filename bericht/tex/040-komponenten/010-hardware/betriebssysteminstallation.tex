\subsubsection{Betriebssysteminstallation}

Dieser Abschnitt erläutert, wie die Grundinstallation auf den Spielgeräten und dem Server
vorgenommen wird.
Für diese Aufgabe war Lennart Weiß verantwortlich.

Der Server ist ein Raspberry Pi Model B+.
Er hat zwei physikalische Netzwerkinterfaces: WLAN und Ethernet.
Aufgrund der geringen Größe lässt er sich leicht überall hin transportieren.
Für ein Spiel benötigt er lediglich eine Stromversorgung, um als internetfähiger Access Point zu
dienen braucht darüber hinaus einen Ethernet Uplink, über dem er per DHCP eine IP Adresse beziehen
kann und Internetzugang hat.
Zum Debugging sind darüber hinaus Bildschirm und Tastatur nützlich.

Auf dem Server ist Raspbian Lite installiert.
Das Gerät lässt sich auch durch einen beliebigen anderen PC austauschen, dass besagte
Netzwerkschnittstellen besitzt und auf dem Debian installiert werden kann (was den meisten gängigen
Prozessorarchitekturen möglich ist).
Raspbian wurde vor allem wegen seiner weiten Verbreitung und Stabilität gewählt.
Für Probleme bei der Serverinstallation kann im Regelfall eine Suche in Debianforen und ähnlichem
eine Lösung gefunden und auf Raspbian übertragen werden.

Auf dem Server ist ein Puppetmaster eingerichtet.
Damit kann genau festgelegt werden, welche Software auf die Clients verteilt werden soll und die
Installation kann automatisiert werden.
Für jeden Client muss auf dem Server ein Zertifikat signiert werden, dieser manuelle Schritt wird
jedoch nur einmal bei der Erstinstallation des Client vorgenommen.
Für die möglichst schnelle Installation der Spielgeräte wurde ein Image mit Raspbian vorbereitet, in
dem eine deutsche Lokalisierung mit englischer Sprache eingestellt wurde und ein Puppet Client
installiert ist.
Außerdem wurde die WLAN SSID und das Passwort eingerichtet und der Client verbindet sich nach dem
Boot Vorgang automatisch per wpa\_supplicant mit dem Server.
Um den Kopiervorgang zu beschleunigen, wurde das Image auf die minimale Größe (ca. 2G) gehalten.
Dadurch kann das Image in wenigen Minuten auf eine SD Karte geschrieben und in den neuen Raspberry
Pi eingesetzt werden.
Außerdem kann die SD Karte eine beliebige Größe haben.

Um den Installationsvorgang zu beenden, muss auf man sich per SSH auf dem Client einloggen.
Per raspi-config wird das Filesystem auf die maximale Größe eingestellt und der Hostname
eingestellt.
Dieser ist ‚\texttt{client-x}‘, wobei \texttt{x} die Client ID bezeichnet, welche vorher eindeutig
an jedes Gerät vergeben wird.
Die Geräte sind entsprechend beschriftet, damit sie identifiziert werden können.
Schließlich werden mit Puppet alle benötigten Komponenten installiert und der Client ist
einsatzbereit.

