\subsubsection{Installation des Projekts}

Dieser Abschnitt erläutert, wie die einzelnen Komponenten des Projekts auf die Geräte verteilt
werden.
Für diese Aufgabe war Lennart Weiß verantwortlich.

Der gesamte Source Code für den verschiedenen Komponenten befindet sich in einem Git repository.
Um die neueste Version des Projekts auf die Geräte zu verteilen, wird auf dem Server ein Skript
aufgerufen, welches zuerst die einzelnen Komponenten kompiliert und die erforderlichen Dateien
installiert.

Dies sind die Komponenten die installiert werden müssen: services-server, services-website, mysql
auf dem Server und services-client, hardware-api und lirc auf dem Client.
Dabei sind mysql und lirc externe Komponenten, die anderen wurden von uns entwickelt.

Für jede Komponente wird eine Systemd Unit definiert und entweder auf dem Server, oder auf allen
Clients verteilt.
Es ist erwünscht, dass die verschiedenen Services nach dem Systemstart automatisch gestartet werden
und mit Systemd können Abhängigkeiten und die richtige Reihenfolge festgelegt werden.
Auf den Clients ist services-client von hardware-api abhängig und letzteres von lircd.
Auf dem Server hängt services-server von services-website und diese wiederum von mysql ab.
Die Prozesse werden also durch das Betriebssystem verwaltet und sollte einer abstürzen, lassen sich
die Logs mit journalctl nachlesen.
Das verwalten vieler Prozesse auf mehreren Geräten lässt sich somit besser handhaben.

