\subsubsection{Installation des Projekts}
\label{sec:installation-des-projekts}

Dieser Abschnitt erläutert, wie die einzelnen Komponenten des Projekts auf die
Geräte verteilt werden. Für diese Aufgabe war Lennart Weiß verantwortlich.

In den vorherigen Abschnitten wurde die Basis für unser Projekt vorgestellt.
Allerdings blieb offen, wie die verschiedenen Komponenten verteilt werden.
Dafür gibt es grundsätzlich zwei Anforderungen: Zum einen muss festgelegt
werden, aus welchen Komponenten das System besteht und welche Installationsschritte
notwendig sind. Zum anderen muss ein Prozess definiert werden, mit dem
die Installation automatisiert vollzogen wird. Letzterer Schritt ist aus
verschiedenen Gründen wichtig. So können durch einen automatisierten
Prozess Flüchtigkeitsfehler ausgeschlossen werden. Aber vor allem die Geschwindigkeit,
mit der simultan die neueste Version unserer Software auf alle Geräte verteilt
werden kann, ist ausschlaggebend. Müsste dies manuell gemacht werden, wäre es
unmöglich, an einem Tag mehrere verschiedenen Versionen auf den Geräten zu
testen, weil nach jeder Änderung die Software neu verteilt werden muss.

Die Systemkomponenten wurden bereits beschrieben. Im folgenden Abschnitt wird
erklärt, wie diese auf welchem System installiert werden. Zuerst müssen externe
Voraussetzungen erfüllt werden. Auf dem Server werden diverse MySQL-Pakete
installiert und auf dem Client einige Lirc-Pakete. Die lirc-Konfiguration muss
angepasst werden und es müssen beim Systemstart zusätzliche Treiber geladen
werden. Die dafür notwendigen Konfigurationsdateien befinden sich im Hardware-Repository.
Die verschiedenen Spielmodi sind LUA-Dateien und befinden sich
im Spielmodus-Repository. Sie sollen auf dem Server nach 
\texttt{/var/lib/spielmodi} kopiert werden. Die Webseite besteht aus einer
Python-Datei und einigen HTML-Seiten. Diese befinden sich in einem Ordner im
Services-Repository und sollen auf dem Server nach 
\texttt{/usr/bin/services-website} kopiert werden. Außerdem müssen mit Pip
einige zusätzliche Pakete installiert werden. Die restlichen Dateien
müssen noch zuvor kompiliert werden. Die daraus resultierenden Binaries 
\texttt{services-client} und \texttt{hardware-api} sollen auf den Client nach
\texttt{/usr/bin} kopiert werden und ebenso \texttt{services-server} auf
den Server. Die Shared Library \texttt{libservices-common.so} wird auf allen
Geräten nach \texttt{/usr/lib} verschoben und auf dem Server wird zusätzlich
\texttt{/usr/libmysqlclient.so.18} in dasselbe Verzeichnis kopiert.

Da nun festgelegt ist was installiert werden soll, erläutern wir den
automatisierten Prozess, um dies zu bewerkstelligen. Auf dem Server befindet
sich ein Skript, das jedes Mal ausgeführt wird, wenn die neueste Version des
Projekts verteilt werden soll. Bevor das Skript aufgerufen wird, muss die
gewünschte Version von Gitlab heruntergeladen werden. Wenn es in der neusten
Version einen schwerwiegenden Fehler gibt, ist es somit möglich auf eine
funktionierende umzusteigen. Als nächstes werden diverse Makefiles aufgerufen,
um die Binaries zu kompilieren. Danach werden die Dateien, die auf dem Server
installiert werden sollen, direkt in die entsprechenden Verzeichnisse kopiert.

Der Installationsprozess für die Clients wird mit einem Orchestrationstool
names Puppet organisiert. Der Server ist der Puppetmaster und in einem
speziellen Ordner befindet sich die Beschreibung des erwünschten Systemzustands
der Client sowie die Dateien, die dafür auf den Client kopiert werden müssen.
Der Vorteil in der Verwendung dieses Werkzeugs ist, dass der erwünschte
Systemzustand der Clients in Gitlab dokumentiert und unter Versionskontrolle
ist. Außerdem kümmert sich nun Puppet im Hintergrund um die Installation und
ist in der Lage, im Falle eines Fehlschlags den bisherigen Installationsprozess
rückgängig zu machen und das System in den bisherigen Zustand zu bringen. Das
Installationsskript auf den Server kopiert automatisch alle für die Clients
benötigten Dateien in das Puppet-Verzeichnis. Die Clients werden nun entweder
innerhalb der nächsten halben Stunde auf die neueste Version umsteigen, oder
der Update-Prozess kann manuell sofort gestartet werden.

Nun müssen die Applikationen noch gestartet und Abhängigkeiten gelöst werden.
Dafür wurde Systemd verwendet. Es ist die Systemstartlösung von Raspbian und
definiert, welche Applikationen in welcher Reihenfolge gestartet werden. Um
den Spielstart zu vereinfachen, sollen die Coding-Tag-Programme bereits starten,
wenn die Geräte mit Strom versorgt werden. Für jedes Programm gibt es eine
Systemd-Unit, die bei jedem Systemstart ausgeführt wird. Damit kann die
richtige Reihenfolge durch Abhängigkeiten definiert werden. Auf dem Server
sind die Webseite und der Services-Server von MySQL abhängig. Auf dem Client
hängt der Services-Client von der Hardware-API ab und diese wiederum von Lirc.
Sollte ein Service abstürzen, können die Fehlermeldungen im Systemlog
nachgelesen werden. Außerdem lassen sich die Programme über \texttt{systemctl}
komfortabel starten oder stoppen.

