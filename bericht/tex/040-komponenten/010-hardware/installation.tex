\subsubsection{Installation des Projekts}

Dieser Abschnitt erläutert, wie die einzelnen Komponenten des Projekts auf die Geräte verteilt
werden.
Für diese Aufgabe war Lennart Weiß verantwortlich.

Der gesamte Source-Code für die verschiedenen Komponenten befindet sich in einem Git-Repository.
Um die neueste Version des Projekts auf die Geräte zu verteilen, wird auf dem Server ein Skript
aufgerufen, welches zuerst die einzelnen Komponenten kompiliert und die erforderlichen Dateien
installiert.

Dies sind die Komponenten die installiert werden müssen: \texttt{services-server},
\texttt{services-website} und \texttt{mysql} auf dem Server sowie \texttt{services-client},
\texttt{hardware-api} und \texttt{lirc} auf dem Client.
Dabei sind \texttt{mysql} und \texttt{lirc} externe Komponenten, die anderen wurden von uns
entwickelt.

Für jede Komponente wird eine Systemd Unit definiert und entweder auf dem Server oder auf allen
Clients verteilt.
Es ist erwünscht, dass die verschiedenen Dienste nach dem Systemstart automatisch gestartet werden
und mit Systemd können Abhängigkeiten und die richtige Reihenfolge festgelegt werden.
Auf den Clients ist \texttt{services-client} von \texttt{hardware-api} abhängig und Letzteres von
\texttt{lircd}.
Auf dem Server hängt \texttt{services-server} von \texttt{services-website} und diese wiederum von
\texttt{mysql} ab.
Die Prozesse werden also durch das Betriebssystem verwaltet und sollte einer abstürzen, lassen sich
die Logs mit \texttt{journalctl} nachlesen.
Das Verwalten vieler Prozesse auf mehreren Geräten lässt sich somit besser handhaben.

Um die benötigten Binaries zu kompilieren, wird die benötigte Version aus dem Git-Repository
gepullt.
Dann wird ein Skript aufgerufen, welches in die verschiedenen Ordner der Komponenten geht und
Makefiles aufruft.
Diese erzeugen dann die erforderlichen Dateien.
Wenn diese auf dem Server benötigt werden, werden sie direkt in \texttt{/usr/bin} oder
\texttt{/usr/lib} kopiert, je nachdem welchen Zweck sie erfüllen.
Die entsprechenden Systemd Units werden dann neu gestartet.
Werden die Dateien auf den Clients benötigt, werden sie in das Puppet-Code-Verzeichnis kopiert.
Wurden diese Dateien noch nicht zuvor verwendet, muss darüber hinaus noch die Konfiguration von
Puppet angepasst werden.
Die Verteilung wird dann über Puppet automatisiert.

Es gibt auch einige Dateien, die verteilt, jedoch nicht kompiliert werden müssen.
Die Webseite wird einfach vom Git-Verzeichnis als Ordner nach \texttt{/usr/bin/services-website}
kopiert.
Die Systemd Unit startet Flask und der Pfad der Flask-Applikation wird exportiert.
Die Spielmodi werden ebenfalls aus Git heruntergeladen und nach \texttt{/var/lib/spielmodi} kopiert.
