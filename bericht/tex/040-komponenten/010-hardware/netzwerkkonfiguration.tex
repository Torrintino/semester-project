\subsubsection{Netzwerkkonfiguration}

Dieser Abschnitt erläutert die Netzwerkinfrastruktur des Projekts.
Für diese Aufgabe war Lennart Weiß verantwortlich.

Die Spielgeräte kommunizieren miteinander über ein WLAN Netzwerk. Die
Netzwerkkonfiguration wird zentral über den Server geregelt. Verglichen mit
einem P2P-Netzwerk oder Infrastruktur Wifi mit Roaming ist der Administrations-
aufwand relativ gering und gleichzeitig sind die Verbindungen stabil, solange
sich die Geräte nah am Server aufhalten.

Um den Server als Access Point zu verwenden wurde das Program hostapd verwendet,
welches den Onboard Wifi Chip des Raspberry Pi verwendet, um ein Netzwerk
aufzuspannen. Diese Chips sind eigentlich dafür gedacht, dass der Raspberry Pi
selber als Client dient und sich zu einem anderen Netzwerk verbindet. Für lange
Distanz und hohe Stabilität wäre eine externe Netzwerkkarte mit Antenne
womöglich besser geeignet. Dennoch hat sich in praktischen Versuchen gezeigt,
dass sich das Coding Tag Netz über mehrere Räume hinweg stabil verwenden lässt
und ist somit für kleine Spielfelder und technische Demonstrationen vollkommen
ausreichend.

Mit dem zuvor beschriebenen Netzwerk allein wäre es bereits möglich gewesen
das Projekt zu realisieren. Allerdings wurden noch weitere Funktionalitäten
hinzugefügt, um zum einen die Projektarchitektur zu verbessern und zum anderen
einen gewissen Komfort zu bieten. Die grundsätzliche Anforderung, die wir uns
selber gestellt haben, war, dass die Clients sich lediglich mit dem Netzwerk
verbinden sollen und die gesamte restliche Konfiguration auf dem Server 
stattfinden soll. Dies reduziert die Notwendigkeit für doppelte Arbeit, da der
Server nur einmal installiert werden muss, Clients jedoch öfter. Außerdem hat
der Server mehr Kontrolle und bei Fehlverhalten liegt die Ursache meistens beim
ihm.

Um DHCP und DNS anbieten zu können, wurde dnsmasq installiert. IP-Adressen
werden aus dem frei wählbarem Adressraum 10.0.0.0/8 gewählt. Dem Server wurde
10.0.0.1 vergeben, die Clients erhalten Adressen aus dem Bereich 10.0.1.1 bis
10.0.1.254. Dies beschränkt die maximale Anzahl an Geräten die am Spiel
teilnehmen können, allerdings wurde anfangs mit einer Spielerzahl von ca.
einem Dutzend gerechnet und falls notwendig lässt sich der Adressbereich einfach
vergrößern. Anfangs gab es im Projekt ein statisches Mapping zwischen der
Client ID und der IP Adresse, z.B. wurde client-2 immer 10.0.1.2 vergeben.
Dies ist nützlich, weil in Logs und dergleichen aus der IP Adresse sofort auf
den Client geschlossen werden kann. Allerdings erwies sich dieses Mapping nicht
als nützlich, weil in der Software nur mit Hostnames gearbeitet wird. Außerdem
war es notwendig, dass die Mappings von ID zu MAC Adressen in einer Datei
gespeichert werden, was zusätzlichen Aufwand bei der Installation bedeutete.
Da die statischen Adressen nur wenig zusätzlichen Nutzen bringen werden nun
Clients beliebige Adressen aus dem verfügbarem Pool zugewiesen.

Im gesamten Projekt wird niemals direkt mit IP Adressen gearbeitet, auf
Software Ebene sprechen sich die Geräte gegenseitig mit Hostnames an. Damit
können in der Netzwerkschicht Änderungen vorgenommen werden, ohne dass die
Software entsprechend geändert werden muss. Die Hostnames werden direkt auf
den Clients gesetzt, dadurch ist bei der Installation eines neuen Gerätes keine
Änderung am Server notwendig. Damit sich die Geräte gegenseitig mit Hostnames
ansprechen können, beantwortet der Server DNS Anfragen und gibt die
erforderliche IP Adresse zurück. Diese kennt der Server, weil er selbst als
DHCP Server jene Adressen vergeben hat.

Die Geräte, sowie der Server, sollten in der Lage sein sich mit dem Internet
zu verbinden, um Pakete herunterladen zu können. Diese Verbindung soll jedoch
optional sein und für keine Komponente im Projekt wird eine Internetverbindung
verwendet, sie ist lediglich im Installationsprozess notwendig. Hierfür wird
die Ethernet Schnittstelle (eth0) des Servers mit einem Uplink verbunden, über
den per DHCP eine IP Adresse und Zugang zum Internet bezogen werden kann. Die
Routingregeln sind so eingestellt, dass eth0 die Default Schnittstelle ist und
lediglich Pakete im 10.0.0.0/8 Netzwerk über die WLAN Schnittstelle gesendet
werden. Damit auch die Client Geräte Internetzugang haben, ist auf dem Server
IP Forwarding aktiviert und es wird NAT mit einfachem Port Forwarding verwendet,
damit Pakete von außen die Clients erreichen, ohne dass die Clients selber eine
IP Adresse in dem Netz brauchen. Damit auch Namen außerhalb des eigenen Netzes
auflösen können, wird der Server als DNS Cache verwendet, der einen externen
Nameserver (8.8.8.8) verwendet für Ziele die er nicht kennt.
