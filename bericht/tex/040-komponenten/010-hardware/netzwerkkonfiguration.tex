\subsubsection{Netzwerkkonfiguration}

Dieser Abschnitt erläutert die Netzwerkinfrastruktur des Projekts.
Für diese Aufgabe war Lennart Weiß verantwortlich.

Der Server ist im wesentlichen ein Wifi Access Point und Router.
Die Clients können sich mit ihm über WLAN verbinden und bekommen über DHCP eine IP-Adresse, behalten
jedoch ihren Hostname.
Der Server fungiert als DNS-Server.
Daher können sich die Geräte untereinander mit dem Hostname ansprechen.
Die IP-Adressen sind nicht statisch und im gesamten Projekt soll generell mit den Hostnames
gearbeitet werden.
Damit wird eine Abstraktion geschaffen, die es ermöglicht, die Netzwerkkonfiguration zu verändern,
ohne dass entsprechende Änderungen an den Applikationen vorgenommen werden müssen.

Die Routerfunktionalität wurde vor allem deswegen implementiert, weil die Clients über das Internet
Pakete und Updates beziehen müssen.
Wenn der Router einen entsprechenden Internetuplink besitzt, leitet er die Pakete der Clients nach
außen durch und umgekehrt.
Durch den Einsatz von NAS benötigt der Server nur eine eigene IP Adresse, aber keine für die
Clients.
Das Arbeiten und Debuggen am Server und den Clients macht es oft notwendig, dass sich der Entwickler
in das WLAN-Netzwerk einloggt.
Daher ist es ein nützlicher Nebeneffekt, dass er weiterhin auf das Internet zugreifen kann, wenn er
mit dem Netzwerk verbunden ist.
Damit der Server sowohl lokale als auch globale URIs auflösen kann, muss er auf dem Client als
Nameserver hinzugefügt werden.
Er fungiert als DNS Cache, Anfragen an externe Ressourcen werden an einen anderen Nameserver
weitergeleitet.
