\newpage
\subsubsection{Schaltungen}

Als Schaltungen wurden lediglich sehr Einfache verwendet. 
\begin{enumerate}
	\item Die wichtigste Schaltung stellt hier die Infrarot-Diodenschaltung dar. Diese besteht aus der Infrarot-Diode (LD-274-3) inklusive Vorwiderstand und einem Verstärkerteil, welcher aus einem Widerstand und einem Transistor (T547C) besteht.\\
	\begin{figure}[h]
		\centering
		\includegraphics[width=0.2 \textwidth]{./040-komponenten/010-hardware/Diodenschaltung.png}
		\caption{Die Dioden-Schaltung}
		\label{fig:Bild2Hardware}
	\end{figure}\\
	Mit dieser Schaltung schafft man es das Signal so zu verstärken, dass man eine Reichweite von ca. fünf Meter hat. Dies ist für ein Indoor-Lasertag ausreichend.\\
	\item Die zweite Schaltung ist zum Auslösen des "Taggers". Sie enthält nur einen Vorwiderstand und einen "Push-Button"\\
	\begin{figure}[h]
		\centering
		\includegraphics[with=0.2 \textwidth]{./040-komponenten/010-hardware/button.png}
		\caption{Die Button-Schaltung}
		\label{fig:Bild3Hardware}
	\end{figure}\\
	Wenn der Button gedrückt wird, wird der Stromkreis geschlossen und der Tagger schießt.
	\item Die dritte Schaltung ist die der Anzeige-LEDs. Es sind drei in unterschiedlichen Farben (grün, gelb und rot), welche jeweils an einem einzelnen GPIO-Pin liegen. Außerdem wurde ein Vorwiderstand verwendet.\\
	\begin{figure}[h]
		\centering
		\includegraphics[with=0.2 \textwidth]{./040-komponenten/010-hardware/LEDschaltung.png}
		\caption{Die LED-Schaltung}
		\label{fig:Bild4Hardware}
	\end{figure}\\
	Die LEDs können einzeln angesteuert werden und so verschiedene Spielzustände anzeigen. Die Funktion wird genauer im Abschnitt 4.1.7 Hardware-API beschrieben.
	\item Der Empfänger ist ein TSOP31256 und an Stromversorgung, Erdung und GPIO-Pin angeschlossen.\\
	Er empfängt auf einer Frequenz von 56kHz, welche dazu führt, dass man eine höhere Reichweite erreicht und weniger Signalstörungen hat.
	
	
	

	
	
	
	
	
\end{enumerate}