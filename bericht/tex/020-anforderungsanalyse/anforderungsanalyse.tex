\section{Anforderungsanalyse}
\label{sec:anforderungsanalyse}

Die Aufgabe war, ein gut spielbares Lasertag nachzubauen mit Regeln, die ein
P2P Netzwerkprotokoll nachahmen. Dies sind sehr wage Anforderungen, in die man
viel hinein interpretieren kann. Somit war den Studenten viel Freiraum
überlassen, ein System zu entwerfen, das dem entspricht.

Bei der Projektvorstellung wurde der Gruppe ein Text vorgelegt, in dem diese
Anforderungen niedergeschrieben sind. In dem war die Rede von 
„Augmented Reality“, daher fällt das Spiel nicht rein virtuell aus, sondern
die Spieler haben echte Geräte, mit denen sie umgehen müssen, und visuelles
Feedback. Wie im Original muss hier eine Sender-Empfänger-Kombination
eingesetzt werden, die an jeden Spieler vergeben wird. Für die Geräte sind
zumindest die Pistolen erforderlich. Es ist jedoch unklar, wo die Sender und
Empfänger angebracht werden. So könnte die Pistole als Sender fungieren und
am Körper werden Sensoren angebracht. Wenn ein Sensor nun ein Signal empfängt,
wurde der entsprechende Spieler getroffen. Man könnte andersherum aber auch
den Sensor an der Pistole befestigen und der Spieler hat einen anderen
getroffen, wenn der Sensor etwas empfängt. In beiden Fällen hat immer nur ein
Spieler die Information des Abschusses, in einem Netzwerkprotokoll müssen
allerdings beide Endpunkte wissen, ob ein Signal angekommen ist. Daher muss
es auch ein Mittel zum Informationsaustausch geben.

Die Forderung nach Spielbarkeit impliziert, dass die Spieler ein Ziel haben
müssen, auf das sie hinarbeiten. Im Lasertag gibt es einen Sieger nach Punkten.
Das steht im Gegensatz zu gewöhnlichen Netzwerkprotokollen, in dem mehrere
Teilnehmer ein gemeinsames Ziel erreichen wollen. Um das Spiel interessant zu
gestalten, muss es ein gewisses Element der Konkurrenz und Herausforderung geben.
Dies ist ein Widerspruch zu Netzwerken, da eine Infrastruktur möglichst
keine Probleme verursachen soll. Außerdem werden dort
Informationen in der Regel nur kopiert, es gibt keine Möglichkeit, Daten zu
klauen oder anderen Teilnehmern Schaden zuzufügen, außer man umgeht die Regeln
oder nutzt Widersprüche und Schwachstellen im Protokoll aus.
Dieses Element der Fairness kann verwendet werden, um den Spielmodus
interessanter zu gestalten. Aber in jedem Fall ist ein Kompromiss notwendig,
da allein schon die Komplexität eines Protokolls die Spielbarkeit unterbindet.

Für den Spielspaß gibt es noch viele weitere und teilweise sehr subtile
Faktoren. So wird bereits eine gewisse Qualität und Ästethik erwartet, in
Bezug auf die Ausrüstung, aber auch auf visuelles Feedback und die
Zuverlässigkeit des Systems. Die Regeln sollten leicht zu lernen sein und
trotzdem eine gewisse Tiefe besitzen, welche den Wiederspielwert erhöht.
Um eine gute Mobilität zu bieten, sollte die Ausrüstung gut handhabbar sein
und vor allem drahtlos. Optimalerweise ist das Spielfeld möglichst groß. Dies
steht jedoch im Widerspruch zur Zuverlässigkeit der Übertragung, die besser ist,
je kürzer die Distanz der drahtlosen Kommunikation ist.

Auch wenn es die Option gibt, mit dem Spielmodus das unfaire Ausnutzen eines
Protokolles zu simulieren, so sollte das eigentliche Spiel mit seinen Regeln
fair sein. Das bedeutet, dass jeder Spieler, der mit demselben Vorwissen spielt,
dieselben Chancen auf Sieg hat und die Regeln nicht umgangen werden können.
Hierbei müssen zwei Fälle unterschieden werden: Ob ein Spieler bewusst versucht,
das Spiel zu manipulieren, oder das System im Hintergrund einen Fehler macht
und somit eine falsche Wertung auftritt. Letzeres sollte natürlich niemals
passieren. Weil aber mit drahtlosen Übertragungstechniken gearbeitet werden
muss, stellen sich komplizierte Fragen in Bezug auf Konkurrenz. Was passiert
etwa, wenn zwei Spieler sich gegenseitig abschießen und der eine Spieler
zwar schneller war, das Signal des Abschusses aber langsamer übertragen wurde?
In dem Fall scheidet womöglich der falsche Spieler aus, wofür aber keiner der
Beteiligten die Schuld trägt.
