\subsection{Netzwerkarchitektur}
\label{sec:architektur-netzwerkarchitektur}

Informationen über Abschüsse müssen eine entsprechende Auswirkung auf den Spielstand haben.
Eine Option dafür wäre, dieselbe Applikation auf jedem Gerät zu installieren und ein P2P-Netzwerk zu
errichten, in dem sich die Geräte konstant über den gegenwärtigen Spielstand austauschen.
Daraus resultiert jedoch ein sehr kompliziertes Protokoll zur Synchronisierung, insbesondere weil
ein Gerät mitten im Spiel die Verbindung verlieren könnte.
Stattdessen wurde eine simplere Client-Server-Architektur gewählt, in der Clients lediglich
Informationen sammeln und diese an den Server weiterreichen.
Der Server wertet sie aus und gibt Informationen über den Spielstand an die Clients zurück.
Durch diese Architektur werden viele potenzielle Logikfehler, die sich durch konkurrente Nachrichten
ergeben, eliminiert.

Die Clients und der Server verwenden zur Kommunikation WLAN. Dies ist die naheliegendste Lösung, da
die Technik zuverlässig ist und Verbindungsprobleme größtenteils durch den TCP/IP-Stack des
Betriebssystems gelöst werden.
Um ein weitaufspannendes Feld zu errichten, wäre es notwendig, mehrere Access Points (AP) zu
errichten, die sich mit Roaming austauschen.
Es wurde allerdings entschieden, dass dies den Rahmen des Projekts sprengen würde und es wurde der
Einfachheit halber entschieden, den Server auch gleichzeitig als AP einzusetzen, womit theoretisch
ein Spielfeld unterstützt wird, dass mehrere Räume umspannt.

Für den Server wurde ebenfalls ein Raspberry Pi (Model B+) verwendet, ein Notebook wäre jedoch
genauso geeignet gewesen.
Der Raspi wurde verwendet, weil es für die Entwicklung praktisch war einen Minicomputer zu haben,
der leicht zu transportieren ist und als WLAN Access Point verwendet werden kann.
