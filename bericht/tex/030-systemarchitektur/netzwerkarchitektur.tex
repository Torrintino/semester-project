\subsection{Netzwerkarchitektur}
\label{sec:architektur-netzwerkarchitektur}

Informationen über Abschüsse müssen eine entsprechende Auswirkung auf den Spielstand haben.
Eine Option dafür wäre, dieselbe Applikation auf jedem Gerät zu installieren und ein P2P-Netzwerk zu
errichten, in dem sich die Geräte konstant über den gegenwärtigen Spielstand austauschen.
Daraus resultiert jedoch ein sehr kompliziertes Protokoll zur Synchronisierung, insbesondere weil
ein Gerät mitten im Spiel die Verbindung verlieren könnte.
Stattdessen wurde eine simplere Client-Server-Architektur gewählt, in der Clients lediglich
den Server informieren, wenn der Spieler abgeschossen wurde.
Der Server wertet die Daten aus und benachrichtigt die Clients über den Spielstatus.
Diese geben dann mit LED's Informationen an den Spieler weiter.
Durch diese Architektur werden viele potenzielle Logikfehler, die sich durch konkurrente Nachrichten
ergeben, eliminiert. Um das Spiel zu starten und zu administrieren wurde eine
Webseite angelegt, die auf dem Server gehosted wird und auf die alle Geräte im
Netzwerk zugreifen können. Über die Webseite wird außerdem der aktuelle
Spielstand angezeigt. Alle für das Spiel relevanten Information, die nicht
rein Spielmodi intern sind, werden in einer Datenbank gespeichert. Die Webseite
benutzt diese um Punktetabelle zu generieren, der Server braucht die Informationen,
um den Spielverlauf zu steuern und die Clients auf den neuesten Stand zu halten.

Die Clients und der Server verwenden zur Kommunikation WLAN. Dies ist die naheliegendste Lösung, da
die Technik zuverlässig ist und Verbindungsprobleme größtenteils durch den TCP/IP-Stack des
Betriebssystems gelöst werden.