\subsection{Die Spielmodi}
\label{sec:architektur-spielmodi}

Es wurde entschieden, dass die Spiellogik möglichst vom Backend getrennt sein soll.
Deswegen wurde sie in Form von LUA-Skripten gekapselt, die vom Server eingebunden werden.
Die Schnittstelle zwischen diesen beiden Komponenten ist ein Informationsaustausch.
Der Server gibt Spielparameter und Abschussinformationen an die LUA-Skripte weiter und die
Spiellogik gibt die resultierenden Auswirkungen auf den Spielstand zurück, damit dieser auf Webseite
und Spielgeräten angezeigt werden kann.

Als zu spielendes Netzwerkprotokoll haben wir eine vereinfachte Version von Bittorrent gewählt.
Diese Wahl fiel unter anderem, da dieses Original auch in der Realität spielerische Elemente bietet.

Tatsächlich wurden mehrere Spielmodi implementiert.
Dies geschah aus verschiedenen Gründen:
\begin{enumerate}
  \item
    Es zwang uns, eine modulare Architektur zu errichten, die völlig unabhängig vom gewählten
    Spielmodus funktioniert.
    Dadurch haben wir eine garantierte Flexibiltät, Änderungen an Spielmodi vornehmen zu können.
  \item
    Die Spielmodi haben eine unterschiedliche Reichhaltigkeit an Funktionen.
    Dies ist nützlich für das Debugging, weil somit gezielt Features getestet und gewisse
    Rahmenbedingungen je nach Spielmodus vernachlässigt werden können.
\end{enumerate}
