\subsection{Die Spielmodi}

Es wurde entschieden, dass die Spielelogik möglichst vom Backend getrennt sein soll.
Deswegen wurde sie in Form von LUA-Skripts gekapselt und wird vom Server eingebunden.
Die Schnittstelle zwischen diesen beiden Komponenten ist ein Informationsaustausch.
Der Server gibt Spielparameter und Abschussinformationen an LUA weiter und die Spielelogik gibt die
resultieren Auswirkung auf den Spielstand zurück, damit dieser auf der Website und den Geräten
angezeigt werden kann.

Als zu spielendes Netzwerkprotokoll haben wir eine vereinfachte Version von Bittorrent gewählt.
Diese Wahl fiel unter anderem, da dieses Original auch in der Realität spielerische Elemente bietet.
So gibt es im Netzwerk etwa Teilnehmer, die möglichst viel Datenvolumen von anderen beziehen wollen,
ohne selber eine entsprechende Upload-Bandbreite anzubieten.
Dieses Element der Fairness sollte eine zentrale Komponente im Spiel sein.

Tatsächlich wurden mehrere Spielmodi implementiert.
Dies geschah aus verschiedenen Gründen:
\begin{enumerate}
  \item
    Es zwang uns eine modulare Architektur zu errichten, die völlig unabhängig vom gewählten
    Spielmodus funktioniert.
    Dadurch habe wir eine garantierte Flexibiltät, Änderungen an Spielmodi vornehmen zu können.
  \item
    Die Spielmodi haben eine unterschiedliche Reichhaltigkeit an Funktionen.
    Dies ist nützlich für das Debugging, weil somit gezielt Features getestet und gewisse
    Rahmenbedingungen je nach Spielmodus vernachlässigt werden können.
\end{enumerate}
