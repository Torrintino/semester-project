\subsection{Die Spielgeräte}

Wie oben bereits beschrieben, war für das Nachbilden von Lasertag eine drahtlose Übertragungsmethode
notwendig, die möglichst zuverlässig ist.
Im realen Lasertag wird dafür Infrarot verwendet und es werden LEDs benutzt, um den Abschuss
darzustellen.
Auch in diesem Projekt fiel die Wahl auf Infrarot.
Es ist im Gegensatz zu Laser nicht schädlich für die Augen und die Komponenten sind weitaus billiger.
Diese sind zum Beispiel LEDs und Sensoren, sowie Open-Source-Treiber für die
Modellierung für Daten.
Es reicht nicht aus einfach nur zu erkennen, dass man getroffen wurde, weil es für die Spielregeln
notwendig ist zu wissen, von welchem Spieler der Schuss kam.
Diese Notwendigkeit ergibt sich daraus, dass zwei Systeme, die über das Netzwerk kommunizieren, dazu
in der Lage sein müssen sich eindeutig zu identifizieren.
Da über Infrarot Daten versendet werden können, wie zum Beispiel bei
Fernbedienungen, ist es somit genau die richtige technische Lösung für das Lasertag.
Die Frequenz des Infrarot-Signals wurde möglichst hoch gewählt, da es so zu höheren Reichweiten und weniger Fehlern bei der Übertragung entstehen. Sie liegt bei 56kHz


Die gegenseitige Identifizierung wurde so gelöst, dass das Spielgerät eines Spielers beim Abschuss
eine ID versendet.
Wenn der Sensor eines Spielers diese ID empfängt, weiß das Gerät, von wem der Schuss kam, und es
weiß ebenso seine eigene ID.
Daher erhält es eine Information der Form: „\emph{Spieler X hat Spieler Y getroffen}“.

Als Gerät fiel die Wahl auf den Raspberry Pi Model Zero W. Die Entscheidung wurde getroffen, weil
ein leichter, stromsparender Computer benötigt wurde, der freiprogrammierbar ist und mit
Infrarotkomponenten erweitert werden kann.
Für den kabellosen Betrieb werden Powerbanks verwendet, um die Geräte mit Strom zu versorgen.
