\subsection{Die Spielgeräte}

Wie oben bereits beschrieben, war für das Nachbilden von Lasertag eine drahtlose Übertragungsmethode
notwendig, die möglichst zuverlässig ist.
Im realen Lasertag wird dafür Infrarot verwendet und es werden LEDs benutzt, um den Abschuss
darzustellen.
Auch in diesem Projekt fiel die Wahl auf Infrarot.
Es ist im Gegensatz zu Laser nicht schädlich für die Augen und es können über eine größere Distanz
zielgerichtet gesendet werden.
Es gibt kostengünstige Komponenten wie LEDs und Sensoren, sowie Open Source Treiber für die
Modellierung für Daten.
Es reicht nicht aus, einfach nur zu erkennen, dass man getroffen wurde, weil es für die Spielregeln
notwendig ist zu wissen, von welchem Spieler der Schuss kam.
Diese Notwendigkeit ergibt sich daraus, das zwei Systeme die über das Netzwerk kommunizieren, dazu
in der Lage sein müssen sich eindeutig zu identifizieren.
Daher war es sehr nützlich, dass über Infrarot Daten versendet werden können, wie zum Beispiel bei
Fernbedienungen.

Die gegenseitige Identifizierung wurde so gelöst, dass das Spielgerät eines Spielers beim Abschuss
eine ID versendet.
Wenn der Sensor eines Spielers diese ID empfängt, weiß das Gerät von wem der Schuss kam und es weiß
ebenso seine eigene ID.
Daher hat es eine Information der Form: Spieler X hat Spieler Y getroffen.

Als Gerät fiel die Wahl auf den Raspberry Pi Model Zero W. Die Entscheidung wurde getroffen, weil
ein leichter, stromsparender Computer benötigt wurde, der freiprogrammierbar ist und mit
Infrarotkomponenten erweitert werden kann.
Für den kabellosen Betrieb werden Powerbanks verwendet, um die Geräte mit Strom zu versorgen.

Als zu spielendes Netzwerkprotokoll haben wir eine vereinfachte Version von Bittorrent gewählt.
Diese Wahl fiel unter anderem, da dieses Original auch in der Realität spielerische Elemente bietet.
So gibt es im Netzwerk etwa Teilnehmer, die möglichst viel Datenvolumen von anderen beziehen wollen,
ohne selber eine entsprechende Upload-Bandbreite anzubieten.
Dieses Element der Fairness sollte eine zentrale Komponente im Spiel sein.

Tatsächlich wurden mehrere Spielmodi implementiert.
Dies geschah aus verschiedenen Gründen:
\begin{enumerate}
  \item
    Es zwang uns eine modulare Architektur zu errichten, die völlig unabhängig vom gewählten
    Spielmodus funktioniert.
    Dadurch habe wir eine garantierte Flexibiltät, Änderungen an Spielmodi vornehmen zu können.
  \item
    Die Spielmodi haben eine unterschiedliche Reichhaltigkeit an Funktionen.
    Dies ist nützlich für das Debugging, weil somit gezielt Features getestet und gewisse
    Rahmenbedingungen je nach Spielmodus vernachlässigt werden können.
\end{enumerate}
