\subsection{Das Spiel}
Zuvor wurde die grundlegende Infrastruktur beschrieben, mit der Informationen über Abschüsse
ausgetauscht werden können.
Das allein reicht für das Spiel noch nicht aus, weil die Spieler eine Möglichkeit brauchen, eine
Übersicht über den Spielstand zu bekommen.
Dafür gibt es zwei Ansätze, die verfolgt wurden:
\begin{itemize}
  \item
    eine Anzeige auf jedem Gerät über den Status des Spielers und
  \item
    eine Anzeige für den Server über den aktuellen Spielstand.
\end{itemize}
Die Anzeige für die Spielgeräte wurde mit LEDs realisiert, die mit verschiedenen Farben anzeigen, ob
ein anderer Spieler getroffen wurde, ob man unverwandbar ist, etc.
Dies ist eine preiswerte Möglichkeit, die durch die GPIO-Anschlüsse der Raspberry Pis unterstützt
wird.
Mehr zu den LEDs ist im \cref{sec:leds} zu finden.

Die Anzeige des Servers wurde mit einer Website umgesetzt, die tabellarisch das Scoreboard anzeigt.
Die Webseite soll auf einem oder mehreren Bildschirmen dargestellt werden, die auf das gesammte
Spielfeld verteilt werden.

In dieser Entscheidung spiegelt sich wider, dass Spieler regelmäßig erfahren wollen, ob sie noch im
Spiel sind oder ob sie einen anderen Spieler getroffen haben.
Daher muss es eine Möglichkeit geben, die entsprechende Anzeige sehr schnell und einfach zu
überprüfen.
Der aktuelle Spielstand muss seltener angesehen werden, weshalb es in Ordnung ist, wenn der Spieler
einige Sekunden braucht, um auf die Webseite zu schauen.

Letztendlich muss auch auf Fairness geachtet werden.
Daher ist es wichtig, dass die Fähigkeit, andere Spieler abzuschießen, limitiert ist.
Dies muss auf mehrere Weisen geschehen:
\begin{enumerate}
  \item
    Die Streuung von Abschüssen muss eingeschränkt sein.
    Dies wurde gelöst, indem Strohhalme über die LEDs gestülpt wurden, die das Aussenden des Signals
    in eine bestimmte Richtung lenken.
  \item
    Damit ein Spieler nicht mehrmals hintereinander getroffen werden kann, gibt es bei vielen
    Spielmodi eine kurze Unverwundbarkeitszeit, nachdem ein Spieler getroffen wurde.
    In dieser Zeit hat der Getroffene die Möglichkeit, sich so zu positionieren, dass er nicht
    erneut getroffen wird.
\end{enumerate}
