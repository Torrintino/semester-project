\subsection{Auswertung}

Es gibt zwei Sichten darauf, wieviel von der geplanten Architektur umgesetzt wurden: zum einen die
Implentierung der einzelnen Komponenten, zum anderen die Integration derselben.
Um eine technische Machbarkeit der Lösung zu demonstrieren, reicht es aus, mit Unit Tests die
Funktionalität der Einzelkomponenten zu beweisen.
Allerdings war als Endprodukt ein Spiel gefordert, wofür es notwendig ist, dass die Integration
reibungslos funktioniert.
Das heißt, ein beliebiger externer Spieler kann das System verwenden, ohne Kenntnisse über deren
interne Funktionsweise zu haben.

Die einzelnen Komponenten, die geplant waren, wurden alle implementiert und zumindest teilweise
getestet.
Die Spielelogikgruppe konnte mit einem Simulator demonstrieren, dass die einzelnen Spielmodi das
Soll-Verhalten aufweisen und die Hardware-Gruppe konnte mit Hilfe eines Treibers demonstrieren, dass
Infrarotabschüsse an den Services-Client durchgereicht und LED Events korrekt angezeigt werden.
Die Services-Gruppe hat die Funktionalität der Webseite demonstriert, indem sie manuell Einträge in
die Datenbank hinzufügten und bestätigten, dass jene Änderungen auf der Webseite übernommen werden. 
Auch grundlegende Client-Server Kommunikation war möglich.

Die Integration der Komponenten ist unvollständig.
Es lässt sich nachweisen, dass der Server mit LUA kommuniziert.
Allerdings lassen Logs darauf schließen, dass Parameter in der falschen Reihenfolge an LUA
weitergegeben werden.
Die Webseite-zu-Server-Integration über die Datenbank funktioniert für einfache Spielmodi, bei
komplexeren bringen jedoch Deadlocks von Mutex-Objekten den Server zum Absturz.
Es kam außerdem zu diversen Fehlern bei Zugriffen des Servers auf die Datenbank.
Der Services-Server und der Services-Client können sich miteinander verbinden und
Abschussinformationen werden ausgetauscht.
Allerdings führt ein Verbindungsverlust des Clients zurzeit noch dazu, dass der zugehörige Spieler sich 
nicht mehr mit dem Spiel verbinden kann.
Diese Problem machen derzeit einen Neustart der Systemd-Units auf allen Geräten notwendig, wenn ein
neues Spiel gestartet wird.
Die Verbindung der Client/Server-Programme mit Systemd ist noch instabil.
Es dauert teilweise Minuten, bis eine Unit gestoppt werden kann.
Die Hardware-API ist in der Lage, zuverlässig Abschussinformationen an den Services-Client
weiterzugeben.
Allerdings werden derzeit keine LED-Events an die Hardware-API gesendet.
Die Integration der Hardware-API mit der darunter liegenden Hardware und Systemd ist stabil und
verursacht keine Probleme.
