\subsection{Auswertung}
\label{sec:auswertung}

Der zuvor beschriebene Systementwurf ist geeignet, um die Anforderungen zu
erfüllen. Es sind Spielgeräte geplant worden, die denen aus Lasertag ähneln,
und mit Infrarot können sich Spieler gegenseitig abschießen. Es sollte mindestens ein
Spielmodus implementiert werden, der an BitTorrent angelehnt ist. Jedoch
wurden weitreichende Anpassungen vorgenommen, um gute Spielbarkeit zu gewährleisten. Es wurden viele weitere
Faktoren bedacht, welche das Spielerlebnis verbessern sollen, etwa die
Möglichkeit, Informationen zum Spielstand zu erlangen, oder die freie
Beweglichkeit im Raum. Jedoch ist nicht alles planmäßig verlaufen und der
Entwurf wurde nicht vollständig umgesetzt.

Zuallererst möchten wir auswerten, wie weit die einzelnen Komponenten
entwickelt wurden. Die Hardware bildet die Basis für das Projekt und ist somit
die Voraussetzung für ein funktionierendes System. Die Gruppe hatte sich zum
Ziel gesetzt, nicht nur ein stabile Infrastruktur zur Verfügung zu stellen,
sondern auch den darüberliegenden Schichten Komfort zu bieten, indem
hardwarenahe Details abstrahiert werden. Dies hat gut funktioniert und hat die
Kommunikation mit den anderen Gruppen erleichert, weil es nur sehr wenig
Schnittstellen gab. 

Die Hardwaregruppe hat außerdem die Installation des 
Projekts übernommen, womit sich die Entwickler voll und ganz auf ihre Arbeit
konzentrieren konnten, ohne sich um die Verteilung der Software kümmern zu
müssen. Das Netzwerk hat sich als zuverlässig erwiesen, jedoch ist die
Reichweite noch ausbaufähig. Die Raspberry Pis haben sich als günstige, aber
sehr effektive Lösung erwiesen und wir konnten mit den Powerbanks die
Geräte komplett ohne externe Kabel realisieren. Allerdings sind sie noch sehr
klobig und die Powerbanks passen nicht rein, wodurch sie die ganze Zeit über
zusätzlich gehalten werden müssen. Insgesamt sind die Geräte nicht sehr schön.

Die LEDs machen das ganze Spiel sehr viel lebendiger, indem sie etwas
optisches Feedback bieten, jedoch müssen sich neue Spieler erst mit den Codes
vertraut machen, was vielleicht etwas unintuitiv ist. Die Infrarotübertragung
hat technisch funktioniert, wir sind uns allerdings noch nicht ganz über ihre
Grenzen klar, weil wir Abschüsse nicht genug getestet haben.

Ziel der Spiellogik-Gruppe war die Erstellung mehrerer Spielmodi im Stil von 
Lasertag, die ein Netzwerkprotokoll für die Spieler erlebbar machen sollten. 
Dabei fiel die Wahl auf BitTorrent als Grundlage für alle Spielmodi, da wir viele 
Analogien zwischen der Funktionsweise von BitTorrent und den allgemeinen
Prinzipien von Lasertag fanden. Es mussten dabei viele Entscheidungen getroffen 
werden, die die Spielmodi vereinfachten und in unserer Vorstellung spielbarer
machten, jedoch damit auch die Unterschiede zwischen den Spielmodi und BitTorrent
vergrößerten. Diese Vereinfachungen bzw. Abwandlungen waren Teil der
Anforderungen. Insofern denken wir, dass wir eine gute Balance zwischen Aspekten 
von BitTorrent und von Lasertag in den Spielmodi gefunden haben. Die Erstellung
mehrerer Spielmodi machte es uns zudem möglich, verschiedene Eigenschaften von 
BitTorrent auf mehrere Spielmodi zu verteilen, um so die Komplexität jedes
einzelnen Spielmodus für die Spieler möglichst klein zu halten.

Leider haben wir die Spielmodi nur mit dem Simulator und nicht im Praxiseinsatz, 
also auf einem echten Spielfeld mit echten Spielern, testen können. Dadurch 
könnten uns Faktoren, die den Spielverlauf in der Praxis beeinflussen und damit
eventuell Anpassungen an den Annahmen und bspw. Punktzahlen der Spielmodi nötig 
machen würden, völlig unbekannt geblieben sein.

Besonders für das Balancing von Parametern wie Punktzahlen und die Zuweisung der
initialen Dateifragmente hätte ein automatisches Testen sinnvoll sein können.
Andererseits hätten auch bei automatischen Tests Rahmenbedingungen bzw. Testfälle 
festgelegt werden müssen, die ohne Daten bzw. Erfahrungen aus der Praxis nur schwer 
realistisch festgelegt hätten werden können.
\newline

Die Service-Gruppe hatte mehrere sehr unterschiedliche Aufgaben zu bewältigen. 
Das waren zum einem die Entwicklung eines Servers, dessen Aufgabe es ist, die Parameter des aktuellen Spieles, 
basierend auf Hardware und Logik-Input zu verwalten. Des Weiteren musste eine Kommunikation zwischen den 
Services-Komponenten und den beiden anderen Gruppen etabliert werden. Zuletzt musste noch ein Weg gefunden werden, 
dem Spieler zu ermöglichen, ein Spiel zu erstellen, zu verwalten und zu starten, sowie den derzeitigen Stand abzufragen.

Im Großteil konnten diese Aufgaben alle bewältigt werden. Der Server selbst kann das Spiel konstant aufrecht erhalten 
und auf Input reagieren. Die Kommunikation zwischen Client und Server, sowie die Funktionen der LEDs sind vollständig implementiert. 
Tests zeigen zwar, dass alle Komponenten korrekt arbeiten, jedoch konnte mangels Parxistest nicht ausgeschlossen werden, dass unerwartete Probleme auftreten.

 Bei Integrationstests zeigte sich jedoch, dass insbesondere Input der Hardwareebene nicht korrekt von der Services-Komponente verarbeitet wird. 
 Eine Kommunikation mit der Logik über LUA konnte hergestellt werden und die Daten konnten korrekt vom Server verarbeitet werden. 
 Da jedoch die Ergebnisse nicht korrekt sind, haben wir einen unlokalisierten Bug in der Aufnahme oder Verarbeitung.
 
Das Problem der Benutzereingabe konnte relativ simpel über eine Webseite gelöst werden. HTML (und CSS) reichte für die 
meisten Funktionen dabei aus, für die anderen benötigten Features könnten JavaScript und Flask ausreichend helfen. 
Zugunsten von Datenpersistenz und einfacher Implementierung simultaner Spiele kommuniziert die Seite nicht (wie die anderen Komponenten) 
schlicht über Google Protocol Buffers, sondern schreibt alle eingegebenen Daten in eine MySQL-Datenbank. Obwohl die Vorteile dieses Systems 
groß sind, hat sich bei dem Versuch der Integration mit dem Server gezeigt, dass viel mehr Probleme auftreten als erwartet, was dazu führte, 
dass es zur Endpräsentation nicht komplett einsatzfähig war.


