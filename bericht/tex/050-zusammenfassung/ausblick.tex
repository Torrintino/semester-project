\subsection{Ausblick}

Bei der Endpräsentation hat sich gezeigt, dass die einzelnen Komponenten des
Projekts noch nicht perfekt zusammenpassen. Allerdings hat die Integration
grundsätzlich bereits funktioniert und mit weiterer Fehlerbehandlung und
Testen lassen sich diese Probleme und allzuviel Arbeit beheben.

Ob ein Spiel Spaß macht lässt sich von einem theoretischen Standpunkt nicht
zeigen. Die beste Methode ist das Spiel zu spielen, Schwachpunkte zu
identifizieren und zu verbessern. Dieser Prozess dauert vermutlich länger und
es braucht einige Mühe bis das Spiel wirklich ausgereift ist.

An vielen Stellen haben Entscheidungen zur Sparsamkeit das Ergebniss wesentlich
geprägt, so haben etwa die Spielgeräte eine mäßige Qualität und die
Spielfeldgröße ist noch recht beschränkt. Sollte man das Spiel aber wirklich
verbessern wollen, sind diese Punkte sehr einfach zu beheben, indem man einfach
mehr Geld ausgibt. So könnte man beispielsweise Antennen für den Server kaufen,
um dessen Reichweite zu verbessern.

Es gibt viele Dinge die wir noch ausprobieren wollten, allerdings entweder aus
Mangel an Zeit oder Geld unterlassen haben. Insbesondere mit dem Feedback für
Spieler sind wir unzufrieden und wir hätten am liebsten jedem ein kleines
Display anstelle der LED's gegeben, z.B. um den Spielstand anzuzeigen, oder
sogar eine Minimap mit Positionsanzeige. Besonders zufrieden sind wir mit der
Architektur der Gruppen gewesen, wir haben das Gefühl, dass sich Änderungen im
Nachhinein leicht hinzufügen lassen können, das lässt für die Zukunft viel
Potential.

Verbesserung sind jedoch nicht nur an dem Endergebniss notwendig, sondern vor
allem am Prozess, mit dem dieses erstellt wurde. Letztendlich ist das wichtigste
Resultat des Projekts die praktische Erfahrung und die Anwendung von Software
Engineering Kenntnissen und gerade dieser Punkt schien auch unsere größte
Schwachstelle zu sein. Der Entwurf und die Architektur, sowie auch die
Implentierung verliefen zufriendenstellend. Doch um ein reifes Produkt zu
erzeugen ist besonders intensives Testen notwendig und bei zusätzlichen
Projekten der Gruppenteilnehmer sollte darauf mehr acht gegeben werden. Es gibt
in dieser Hinsicht einiges worin wir uns verbessern könnten. Es gibt kaum
Dokumentation und es wäre ideal gewesen, wenn wir Anforderungen und dergleichen
schon vor der Implementierung dokumentiert hätten, um daraus Testfälle
abzuleiten. Vor allem aber die Organisation der Tests hat komplett gefehlt.
Zumindest an der Absprache der Gruppen kann man noch arbeiten, um
Integrationstests besser voranzutreiben. Vielleicht hätten wir sogar soweit
gehen sollen, eine Person dediziert als Tester einzusetzen. Dies wäre
besonders bei der Servicesgruppe angebracht, weil dort die größte Menge an
komplexer Software entstanden ist, die mit vielen verschiedenen Komponenten
kommunizieren muss.

Schlussendlich wollen wir bei den Projektorganisatoren bedanken, für viele
hilfreiche Ratschläge, welche wesentlich zur ausgereiften Architektur
beigetragen haben. Wir möchten uns außerdem dafür bedanken, dass wir die
Räume des Lehrstuhls intensiv benutzen durften und wir viel Freiraum bekommen
haben, um unsere Ideen zu verwirklichen. Wir haben vieles aus dem Projekt 
gelernt, seien es neue Technologien oder Zusammenarbeit und es war ein
großartiges Gefühl, als wir zum ersten Mal alle Komponenten zusammenbringen
konnten.
